The final day! My talk is in the RL session this morning so I unfortunately didn't make it to the keynote.

\subsection{Reinforcement Learning}

First, some RL.

\subsubsection{Diversity-Driven Hierarchical RL~\cite{song2018diversity}}

Paper by Yuhang Song, Jianyi Wang, Thomas Lukasiewicz, Zhengua Xu, and Mai Xu. \\

Code at \url{github.com/YuhangSong/DEHRL} \\

{\bf Focus:} Hierarchical RL (HRL) \\

$\ra$ Idea: recombine sequences of basic actions to form sub-policies~\cite{sutton1999between}. \\

HRL can speed up learning, transfer, add explainability. \\

{\bf Game of Focus:} Overcooked. Cooking game where agents move around a grid and place objects in different locations subject to a timer. Need to place ingredients on certain objects (bread in toaster, etc.) with the goal of making food based on (randomly generated) orders given. \\

Task Features:
\begin{itemize}
    \item Primitive actions
    \item Abstract goals
    \item Sparse extrinsic reward (only received when agent collects the right ingredients based on the order
\end{itemize}

Using HRL on Overcooked: can use subpolicies to move agent in each direction, then higher level policies to move agent to each destination where ingredients can be collected/placed. \\

{\bf Main Idea:} Exploit and learn diversity driven policies for use in HRL. Comes from the following assumption:

\begin{assumption}
Learning different sub-policies is beneficial (popular in literature).
\end{assumption}

Their extension:
\begin{assumption}
With limited capability, learning sub policies far away from each one as possible is even better.
\end{assumption}

So, with this assumption, their solution:
\begin{itemize}
    \item Diversity of sub-policies can be measured by the {\bf distance} between the resulting states of sub-policies.
    
    \item Diversity-driven solution:
    
    $\ra$ Transition model memorize resulting states of different sub-policies
    
    $\ra$ Intrinsic reward generated based on if a sub-policy leads to a state that is far away from the resulting states if choosing another sub-policies.
\end{itemize}

Form a neural network that trains the many aspects of the above solution (models, policies, and so on). \\

Experiments (in Overcooked):
\begin{itemize}
    \item Sub-policy discovery at level 1 goes to the ``five most diverse/useful states" (out of around 600).
    \item At level 2, sub-policies fetch different ingredients at the four corners.
\end{itemize}

Experiments (in Minecraft) with no reward/supervision: goal is to break a block, build a block, or jump on a block.

$\ra$ Random policies can't do anything, but their HRL approach can actually build some nice structures. \\



\dnote{I'm up!}

\spacerule
\subsubsection{Towards Better Interpretability in DQN~\cite{annasamy2018towards}}

Paper by Raghuram Mandyam Annasamy and Katia Sycara. \\


{\bf Goal:} Interpretability! \\

$\ra$ Important for Deep RL since most neural nets used are treated as a black box, but naturally these systems will be deployed in crucial areas soon. \\

Lots of examples of seeking interpretability in supervised learning. But, it's harder in Deep RL\\

Proposed Method: ``interpretable"-DQN. Keys initialized randomly and learned via backprop. \\

Four loss functions:
\begin{itemize}
    \item Bellman Error (for $Q$ Learning)
    \item Distributional Error (force good representations/robustness \dnote{I think, missed this one})
    \item Reconstruction Error (force interpretable)
    \item Diversity Error (force attention)
\end{itemize}

Intuition:
\begin{itemize}
    \item Use key-value pairs to enforce interpretability.
    \item Keys act like cluster-centers
    \item Can then evaluate/visualize these cluster centers through things like t-sne.
    \item Pass keys through conlutional network to generate canonical states
\end{itemize}

Evaluation:
\begin{itemize}
    \item ``Agreement metric" used for evaluating the approach.
    \item Induce new distribution in image space.
    \item Random agreement will be around $1/|A|$.
    \item In Pac-Man they find 30\% agreement vs the random (which is roughly 10\%).
\end{itemize}

Examples of memorization: Create small permutations to existing states and test what the agent does. \\

Provide an interface for visualizing the reconstructions when small changes are made to the image. Find their i-DQN can do a good job of ignoring the perturbations. \\

{\bf Problem:} Lots of Deep RL methods overfit!~\cite{cobbe2018quantifying,zhang2018dissection}. Can use i-DQN to explore this a bit. 



\spacerule
\subsubsection{On RL for Full-Length Game of Starcraft~\cite{pang2018reinforcement}}

Paper by Zhen-Jia Pang, Ruo-Ze Liu, Zhou-Yu Meng, Yi Zhang, Yang Yu, and Tong Lu. \\

Why Starcraft?
\begin{itemize}
    \item Game is ideal for RL
    \item Most successful real time strategy of all time
    \item Really hard for human and AI to play
    \item If SC is solved, most games can be solved
\end{itemize}

\ddef{Starcraft}{Starcraft is a game where you control a base, units, and manage resources. You must build up a base and good units to destroy the other enemies.}

Difficulties: Starcraft can be multi-agent, is partially observable, and requires both macro level reasoning and low level control over many units. \\

Main focus: massive state-action space {\it and} long horizon until receiving reward. So, how can we solve this? \\

Approach:
\begin{itemize}
    \item Low-level abstraction: build a building (select unit $\ra$ build building $\ra$ go back to work), or produce a unit (select building, etc.).
    
    $\ra$ Can be inferred from mirroring human demonstrations, gives rise to macro-actions.
    
    \item High-level abstraction: Learn high level policies over these macro-actions (the ``low-level" abstractions).
\end{itemize}

Experiments:
\begin{itemize}
    \item Map: simple64
    \item Enemy: Terrain (built in AI)
    \item Agent: Protoss
    \item Fixed types of units/buildings for agent (but not enemy).
    \item High level policy that is decomposed into:
    \begin{enumerate}
        \item a ``base" policy responsible for handling base building and resource management.
        \item a ``battle" policy responsible for dealing with battle/focused on individual control.
    \end{enumerate}
\end{itemize}

Conclusions: 1) Investigate hierarchical architecture for SC, 2) Simple yet effective training algorithm, and 3) Achieves SOTA results on SC. \\


\spacerule

\subsection{Reasoning under Uncertainty}

Next some reasoning under uncertainty.

\subsubsection{Collecting Online Learning of GPs in Multi-Agent Systems}

Paper by Nghia Hoang, Quang Minh Hoang, Bryan Kian Hsiang Low, and Jonathon How. \\

Collecting Learning Motivation:
\begin{itemize}
    \item Local agents that can upload local statistics
    \item Old approach: Central server combine and broad cast
    $\ra$ But, limitation of typical approaches: centralized risk of failure and communication/computational bottleneck.
    
    \item Their approach fixes these shortcomings by removing central server. 
    
    $\ra$ No centralized risk of failure and no computational/communication bottlenecks.
\end{itemize}

Challenges: decentralization! Not clear how to do this (especially with resource constraints). \\

{\bf Goal:} Develop efficient local model representation for online update with data. \\

Gaussian Process (GP prediction is {\it not efficient}:
\begin{itemize}
    \item Computation is cubis
    \item Representation is quadratic
    \item Update is cubic
\end{itemize}

Solution: exploit sparse coding $\bm{u} = u(Z)$ is distributed by a standard GP. \\

Q: How can agent share models without communicating raw data? \\

A: Do local (bayesian/GP) updates, then share a representation that merges these updates. Main idea is to exploit additive structure to make this shared representation from a few messages.\\


Experiments: test on (real-world) traffic data set, characterizes different traffic phenomena. \\

$\ra$ Finding: they can reduce error with more data efficiently. 


\spacerule


\subsubsection{Weighted Model Ingeration using Knowledge Compilation}

Paper by Pedro Zuidberg Dos Martiers, Anton Dries, and Luc de Raedt. \\

{\bf Setting:} Probabilistic inference.  Goal is to combine the best of continuous and discrete inference \\

Knowledge compilation: take a boolean formula, compile it offline line so that you can do quick inference online. \\

New Method that can handle all of 1) knowledge compilation, 2) inference of density functions, 3) exact and approximate inference, and 4) can handle poly-nomial or non-linear constraints. \\

{\bf Contribution:} Can handle probability density functions while applying state of the art knowledge compilation techniques, along with two new solvers: Symbo and Sampo. \\

Symbo does: 1) abstract theory, 2) compile formula, 3) to arithmetic circuit, 4) label leaves, 5) evaluate, 6) multiply by weight of continuous variables, 7) integrate. \\

Sampo does: approximate Monte Carlo inference for doing approximate probabilistc inference with linear time dependency.  First sampling based algorithm for ``WMI"\\

$\ra$ Both samplers beat state of the art.


\dnote{Back to RL}

\subsubsection{Off-Policy Deep RL by Bootstrapping the Covariate Shift}

Paper by Carles Gelada and Marc Bellemare. \\

TD Policy Evaluation with Linear models: samples $s \sim d_\pi, a \sim \pi, r = R(s,a), s' \sim P(\cdot \mid s, a)$. \\

Linear value approximation $V_k(s) = \phi(s) \theta_k$. \\

Weight update:
\[
\theta_{k+1} \la \theta_k + \alpha(r + \gamma V_k(s') - V_k(s)) \phi(s).
\]
Can treat the above weight update as application of an operator. \\

Idea: also do {\it projection} operator this way: $\Pi_{d_\pi} x$. Well known result that Bellman operator converges. \\

{\bf Off Policy Problem:} End up with a {\it different} operator that does not lead to convergence. Well know difficulty from Baird (Baird's counter example)~\cite{baird1995residual}. \\

\ddef{Covariate Shift}{Sample from another distribution $\mu$, induces the covariate shift: $\frac{d_\pi(s)}{d_\mu(s)}$.}

Yields an update rule defined by this ratio, gives an operator:
\[
C_{k+1} = \Pi_{d_\mu} Y c_k.
\]
which {\it does} converge. But, it might converge to bad behaviour. \\

Idea: Can treat this as a projection into a simplex, which lets them translate the above new update rule to cooperate with neural networks. \\

Interpretation of the discount here: discounted of value function relxas the operator $P_\pi$ to $\gamma P_\pi$. Similarly, treat operator $Y$ as relaxing $P_\pi$ to $\gamma P_\pi + (1-\gamma) ed_\mu$. \\

{\bf Result:} $N$ step contraction factor under the discounted operator $\gamma Y$. 

\begin{theorem}
For any $n$, for any $\gamma$, can find an operator $Y$ that converges to a fixed point.
\end{theorem}

Experiment with the C51 agent (the distr. RL agent). Evaluate in ``extreme off policy tasks"), like seaquest. \\

\dnote{Back to reasoning under uncertainty.}




\spacerule
\subsubsection{Compiling Bayes Net Classifiers into Decision Graphs~\cite{shih2019compiling}}

Paper by Andy Shih, Arthur Choi, and Adnan Darwiche. \\

{\bf Goal:} Compile a (Bayes Net) classifier into a decision graph to better explain a classifier. \\


Case study: win95pts. \\

$\ra$ consider a printer with some symptons (slow printing, low toner, etc.). \\

Two kinds of explanations: can you fix a set of features such that the remaining features are rirrelevant, or can we remove erreoneous features. \\

{\bf Compilation:}
\begin{enumerate}
    \item Insight 1: recursively decompose into subclassifiers.
    \item Insight 2: Identify subclasiffiers that are equivalent to save computation time.
\end{enumerate}

Q: How can we. given two subclassifiers $B_1$ and $B_2$, identify  if $\forall_{v \in V} B_1(v) = B_2(v)$. \\

A: Goal is to compute $\Pr(C = 1 \mid v) \geq T$, which we can rewrite as a linear inequality. Problem becomes separating points on a line, so we binary search in time $O(|V|)$. \\

Experiments: use classifiers from literature that we can compile successfully into small ODDs. Try win95pts, Andes, cpcs54 $\ra$ they can all be merged and reduced quite dramatically. \\

Scaled this to networks with over 40 features. \\

{\bf Takeaways:}
\begin{itemize}
    \item Classifiers have an underlying decision function
    \item Decision functions can be compiled into an ODD
    \item Explanations and verification are efficient on ODDs (which are otherwise intractable on Bayesian network classifiers).
\end{itemize}